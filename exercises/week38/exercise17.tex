% !TEX TS-program = pdflatex
% !TEX encoding = UTF-8 Unicode

% This is a simple template for a LaTeX document using the "article" class.
% See "book", "report", "letter" for other types of document.

\documentclass[12pt]{article} % use larger type; default would be 10pt

\usepackage[utf8]{inputenc} % set input encoding (not needed with XeLaTeX)

%%% Examples of Article customizations
% These packages are optional, depending whether you want the features they provide.
% See the LaTeX Companion or other references for full information.

%%% PAGE DIMENSIONS
\usepackage{geometry} % to change the page dimensions
\geometry{a4paper} % or letterpaper (US) or a5paper or....
% \geometry{margin=2in} % for example, change the margins to 2 inches all round
% \geometry{landscape} % set up the page for landscape
%   read geometry.pdf for detailed page layout information

\usepackage{graphicx} % support the \includegraphics command and options

\usepackage[parfill]{parskip} % Activate to begin paragraphs with an empty line rather than an indent

%%% PACKAGES
\usepackage{booktabs} % for much better looking tables
\usepackage{array} % for better arrays (eg matrices) in maths
\usepackage{paralist} % very flexible & customisable lists (eg. enumerate/itemize, etc.)
\usepackage{verbatim} % adds environment for commenting out blocks of text & for better verbatim
\usepackage{subfig} % make it possible to include more than one captioned figure/table in a single float
\usepackage{amsmath, amssymb}

%%% HEADERS & FOOTERS
\usepackage{fancyhdr} % This should be set AFTER setting up the page geometry
\pagestyle{fancy} % options: empty , plain , fancy
\renewcommand{\headrulewidth}{0pt} % customise the layout...
\lhead{}\chead{}\rhead{}
\lfoot{}\cfoot{\thepage}\rfoot{}

%%% SECTION TITLE APPEARANCE
\usepackage{sectsty}
\allsectionsfont{\sffamily\mdseries\upshape} % (See the fntguide.pdf for font help)
% (This matches ConTeXt defaults)

%%% ToC (table of contents) APPEARANCE
\usepackage[nottoc,notlof,notlot]{tocbibind} % Put the bibliography in the ToC
\usepackage[titles,subfigure]{tocloft} % Alter the style of the Table of Contents
\renewcommand{\cftsecfont}{\rmfamily\mdseries\upshape}
\renewcommand{\cftsecpagefont}{\rmfamily\mdseries\upshape} % No bold!

\newcommand{\Dt}{\Delta t}
%%% The "real" document content comes below...

\title{Leapfrog scheme}
\author{Nina Kristine Kylstad, Ingeborg Sauge Torpe}
%\date{} % Activate to display a given date or no date (if empty),
         % otherwise the current date is printed 

\begin{document}
\maketitle

\section*{Exercise 17}

We want to analyze the Leapfrog scheme by looking at the exact solution of the discrete equation. We consider the case where a is constant and $b = 0$, giving

\begin{equation}
u'(t) = -au(t)\,,\;u(0) = I
\label{ODE}
\end{equation}
where $I$ is some initial condition. We assume that the exact solution of the discrete equations is on the form 
\begin{equation}
u^n = A^n
\label{dis}
\end{equation}
The leapfrog scheme for (\ref{ODE}) can be written as
\begin{equation}
u^{n+1} = u^{n-1} - 2a\Delta tu^n
\label{LF}
\end{equation}
We insert (\ref{dis}) into (\ref{LF}):
\begin{align*}
A^{n+1} &= A^{n-1} - 2a\Delta t A^n\\
\Rightarrow A^2 &= (2a\Delta t)A - 1  = 0\\
\Rightarrow A &= \frac{-2a\Delta t \pm \sqrt{(2a\Delta t )^2 - 4\cdot1\cdot-1}}{2}\\
&= -a\Delta t \pm  \sqrt{(a\Delta t )^2 +1}
\end{align*}
We can see that the governing polynomial for A has two roots, $A_1$ and $A_2$. This means that $A^n$ is a linear combination of $A_1$ and $A_2$, 
\begin{equation}
A^n = C_1A_1^n + C_2A_2^n
\end{equation}
where $C_1$ and $C_2$ are constants to be determined. The root $A_1$ is negative, and can therefore cause oscillations.

To find the constants $C_1$ and $C_2$, we use the initial condition $I$, and the value we obtain for $u^1$ by using the Forward Euler scheme:
\[u^1 = u^0 - \Dt au^0 = I(1 - \Dt a)\]
To simplify, we let $x = \Dt a$. The equation for $A^n$ then becomes:
\begin{equation}
A^n = C_1(-x - \sqrt{x^2 +1})^n + C_2(-x + \sqrt{x^2 +1})^n
\end{equation}

We can now find $C_1$ and $C_2$.
\begin{align*}
A^0 &= C_1 + C_2 = I\\
\Rightarrow C_1 &= I - C_2\\
A^1 &= C_1(-x - \sqrt{x^2 +1}) + C_2(-x + \sqrt{x^2 +1}) = I(1 - x)\\
\Rightarrow C_2 &= \frac{I(1 + \sqrt{x^2 +1})}{2\sqrt{x^2 +1}}
\end{align*}

To test how the roots $A_1$ and $A_2$ affect the numerical solution, we find the values of $C_1A_1^n$ and $C_2A_2^n$ for increasing values of n. This is done in the program \texttt{dc\_leapfrog\_analysis.py} A sample of the output is as follows:

\begin{verbatim}
1x-193-157-247-37:Downloads ninakylstad$ python dc_leapfrog_analysis.py
n = 0
0.029289321903        0.070710678097        0.100000000000
n = 1
-0.029583679552        0.070007106761        0.099004983375
n = 2
0.029880995494        0.069310535961        0.098019867331
n = 3
-0.030181299462        0.068620896042        0.097044553355
n = 4
0.030484621484        0.067938118040        0.096078943915
n = 5
-0.030790991892        0.067262133681        0.095122942450
n = 6
0.031100441322        0.066592875367        0.094176453358
n = 7
-0.031413000718        0.065930276174        0.093239381991
n = 8
0.031728701336        0.065274269843        0.092311634639
n = 9
-0.032047574745        0.064624790777        0.091393118527
n = 10
0.032369652831        0.063981774028        0.090483741804
	.					.					.
	.					.					.
	.					.					.
	.					.					.
n = 391
-1.461411232877        0.001417169765        0.002004050106
n = 392
1.476098413941        0.001403068924        0.001984109474
n = 393
-1.490933201156        0.001389108386        0.001964367255
n = 394
1.505917077964        0.001375286756        0.001944821475
n = 395
-1.521051542715        0.001361602651        0.001925470178
n = 396
1.536338108818        0.001348054703        0.001906311429
n = 397
-1.551778304892        0.001334641557        0.001887343314
n = 398
1.567373674916        0.001321361872        0.001868563934
n = 399
-1.583125778390        0.001308214319        0.001849971412
n = 400
1.599036190484        0.001295197585        0.001831563889
\end{verbatim}

As we can see from the output, the root $A_1$ begins oscillating right from the beginning. For low values of $n$, we see that $C_1A_1^n$ is quite small, and therefore does not affect the solution much. However, it becomes larger as $n$ becomes larger, and as we can see from the last 10 values of $n$ shown here, $C_1A_1^n$ becomes considerably larger than both $C_2A_2^n$, and the exact analytical solution. Because of this, the numerical solution oscillates more and more with larger $n$. 









\end{document}
