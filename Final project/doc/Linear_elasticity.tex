%%
%% Automatically generated ptex2tex (extended LaTeX) file
%% from Doconce source
%% http://code.google.com/p/doconce/
%%




%-------------------------- begin preamble --------------------------
\documentclass[twoside]{article}



\usepackage{relsize,epsfig,makeidx,amsmath,amsfonts,fullpage,color,mdframed}
\usepackage[latin1]{inputenc}
%\usepackage{minted} % packages needed for verbatim environments


% Hyperlinks in PDF:
\usepackage[%
    colorlinks=true,
    linkcolor=black,
    %linkcolor=blue,
    citecolor=black,
    filecolor=black,
    %filecolor=blue,
    urlcolor=black,
    pdfmenubar=true,
    pdftoolbar=true,
    urlcolor=black,
    %urlcolor=blue,
    bookmarksdepth=3   % Uncomment (and tweak) for PDF bookmarks with more levels than the TOC
            ]{hyperref}
%\hyperbaseurl{}   % hyperlinks are relative to this root


% Tricks for having figures close to where they are defined:
% 1. define less restrictive rules for where to put figures
\setcounter{topnumber}{2}
\setcounter{bottomnumber}{2}
\setcounter{totalnumber}{4}
\renewcommand{\topfraction}{0.85}
\renewcommand{\bottomfraction}{0.85}
\renewcommand{\textfraction}{0.15}
\renewcommand{\floatpagefraction}{0.7}
% 2. ensure all figures are flushed before next section
\usepackage[section]{placeins}
% 3. enable begin{figure}[H] (often leads to ugly pagebreaks)
%\usepackage{float}\restylefloat{figure}
\usepackage{parskip}

\newcommand{\inlinecomment}[2]{  ({\bf #1}: \emph{#2})  }
%\newcommand{\inlinecomment}[2]{}  % turn off inline comments
\newcommand{\dutt}{\frac{\partial^2 \mathbf{u}}{\partial t^2}}
\newcommand{\duxx}{\frac{\partial^2 \mathbf{u}}{\partial x^2}}
\newcommand{\dut}{\frac{\partial \mathbf{u}}{\partial t}}
\newcommand{\dux}{\frac{\partial \mathbf{u}}{\partial x}}
\newcommand{\dtt}{\Delta t^2}
\newcommand{\dxx}{\Delta x^2}
\newcommand{\dt}{\Delta t}
\newcommand{\dx}{\Delta x}
\newcommand{\wt}{\tilde{\omega}}
\newcommand{\unp}{\mathbf{u^{n+1}}}
\newcommand{\un}{\mathbf{u^{n}}}
\newcommand{\unm}{\mathbf{u^{n-1}}}
\newcommand{\io}{\int\limits_\Omega}
\newcommand{\sumjn}{\sum^N_j}

% insert custom LaTeX commands...

\makeindex

\begin{document}
%-------------------------- end preamble --------------------------





% ----------------- title -------------------------

\begin{center}
{\LARGE\bf Linear elasticity - Finte elements}
\end{center}


\vspace{0.5cm}
\begin{center}
% List of all institutions:
\centerline{INF5620 - Numerical methods for partial differential equations}
\end{center}
% ----------------- end author(s) -------------------------



% ----------------- date -------------------------


\begin{center}
\today
\end{center}

\vspace{1cm}


\section{Mathematical problem}
We consider the general time-dependent equations for linear elasticity,

\begin{align}
\rho \mathbf{u}_{tt} &= \nabla\cdot\sigma + \rho \mathbf{b},     \label{eq:lin:elast}\\
\sigma &= \sigma(\mathbf{u}) = 2\mu\epsilon(\mathbf{u}) + \lambda tr(\epsilon(\mathbf{u}))I     \label{sigma}\\
\mathbf{u}(\mathbf{x}, 0) &= \mathbf{u_0}       \label{ic:u}\\
\mathbf{u}_t(\mathbf{x},0) &= \mathbf{v_0}     \label{ic:ut}\\
\mathbf{u}(\mathbf{0}, t) &= \mathbf{g}        \label{bc:dir}
\end{align}

where $\epsilon(\mathbf{u}) = \frac{1}{2}\left(\nabla \mathbf{u} + \nabla \mathbf{u}^T\right)$ is the strain tensor, $\mu$ and $\lambda$ are the Lam\'e parameters, and I is the identity tensor. We will consider Dirichlet boundary conditions, as shown in \eqref{bc:dir}. The initial conditions are given by \eqref{ic:u} and \eqref{ic:ut}.


\section{Discretization}

\subsection{Finite Difference in time}

We begin by discretizing the time derivative, using a centered difference approximation:
\begin{equation}
\dutt  \approx \frac{\unp - 2 \un + \unm}{\dtt}
\end{equation}

We insert this approximation into \eqref{eq:lin:elast} and evaluate at time $t=t_n$:
\begin{equation}
\rho \frac{\unp - 2 \un + \unm}{\dtt} = \nabla\cdot\sigma(\mathbf{u}^n) + \rho \mathbf{b}^n
\end{equation}

For simplicity, we let $\sigma^n = \sigma(u^n)$. We solve for $\unp$:
\begin{align}
&\rho (\unp - 2 \un + \unm) = \dtt \nabla\cdot\sigma^n +  \rho \mathbf{b}^n  \nonumber\\
&\unp = 2 \un - \unm + \frac{\dtt}{\rho} \nabla\cdot\sigma^n + \dtt \mathbf{b}^n
\end{align}

We now have a discretization in time.


\subsection{Finite Elements in space}
We use a Galerkin method to find the variational form of \eqref{eq:lin:elast}, by introducing a test function $\mathbf{v} \in V$:

\begin{equation}
\io \unp \cdot \mathbf{v} dx = 2\io \un \cdot \mathbf{v} dx - \io \unm \cdot \mathbf{v} dx + \frac{\dtt}{\rho} \io (\nabla\cdot\sigma^n) \cdot 
\mathbf{v} dx + \dtt \io \mathbf{b}^n \cdot \mathbf{v} dx
\end{equation}

We need to integrate the $\io (\nabla\cdot\sigma^n) \cdot v dx$ -term by parts, as it contains second derivatives. We get
\begin{equation}
\io (\nabla\cdot\sigma^n) \cdot \mathbf{v} dx = - \io \sigma^n : \nabla \mathbf{v} dx + \int\limits_{\partial\Omega} (\sigma^n\cdot \mathbf{n})\cdot \mathbf{v} ds
\end{equation}
where $\sigma^n : \nabla \mathbf{v}$ is a tensor inner product. The term $\int\limits_{\partial\Omega} (\sigma^n\cdot n)\cdot \mathbf{v} ds$
vanishes because of the Dirichlet boundary condition. We are now left with 

\begin{equation}
\io \unp \cdot \mathbf{v} dx = \io (2\un - \unm  + \dtt \mathbf{b}^n) \cdot \mathbf{v} dx - \frac{\dtt}{\rho} \io \sigma^n : \nabla \mathbf{v} dx
\label{gen:scheme}
\end{equation}

This is the general scheme. We need to implement the initial conditions in order to get a special scheme for the first time step:

\begin{align}
&\mathbf{u}^0 = \mathbf{u_0} \nonumber\\
&\mathbf{u}_t(\mathbf{x},0) = \mathbf{v_0} \Rightarrow \frac{\unp - \unm}{2\dt}\approx \mathbf{v_0}\nonumber\\
&\Rightarrow \mathbf{u}^{-1} = \mathbf{u}^{1} - 2\dt \mathbf{v_0}\nonumber
\end{align}

We put this expression for $\mathbf{u}^{-1}$ into \eqref{gen:scheme}:

\begin{align}
2\io \mathbf{u}^{1} \cdot \mathbf{v} dx &= \io (2\mathbf{u}^{0} - 2\dt \mathbf{v_0}  + \dtt \mathbf{b}^0) \cdot \mathbf{v} dx - \frac{\dtt}{\rho} \io \sigma^0 : \nabla \mathbf{v} dx\nonumber\\
\io \mathbf{u}^{1} \cdot \mathbf{v} dx &= \io (\mathbf{u_0} - \dt \mathbf{v_0}  + \frac{\dtt}{2} \mathbf{b}^0) \cdot \mathbf{v} dx - \frac{\dtt}{2\rho} \io \sigma(\mathbf{u}_0) : \nabla \mathbf{v} dx
\label{first:step}
\end{align}

We are now ready to express \eqref{eq:lin:elast} as a variational problem. We let $u=\unp$ represent the unknown at the next time level:
\vspace{0.5cm}
\begin{mdframed}
\begin{align}
&a_0(\mathbf{u},\mathbf{v}) = \io \mathbf{u} \cdot \mathbf{v} dx \\
&L_0(\mathbf{v}) = \io \mathbf{u_0} \cdot \mathbf{v} dx \\
&a_1(\mathbf{u},\mathbf{v}) = \io \mathbf{u} \cdot \mathbf{v} dx \\
&L_1(\mathbf{v}) = \io (\mathbf{u_0} - \dt \mathbf{v}_0  + \frac{\dtt}{2} \mathbf{b}^0) \cdot \mathbf{v} dx - \frac{\dtt}{2\rho} \io \sigma(\mathbf{u_0}) : \nabla \mathbf{v} dx \\
&a(\mathbf{u},\mathbf{v}) = \io \mathbf{u} \cdot \mathbf{v} dx \\
&L(\mathbf{v}) = \io (2\un - \unm  + \dtt \mathbf{b}^n) \cdot \mathbf{v} dx - \frac{\dtt}{\rho} \io \sigma^n : \nabla \mathbf{v} dx\\
&L(\mathbf{v}) =  \io (2\un - \unm  + \dtt \mathbf{b}^n) \cdot \mathbf{v} dx - \frac{\dtt}{\rho} \io (2\mu\epsilon(\mathbf{u}) + \lambda tr(\epsilon(\mathbf{u}))I) : \nabla \mathbf{v} dx
\end{align}
\end{mdframed}

With this formulation, we could implement the problem in FEniCS.

However, it is possible to go further, in order to make the implementation more efficient. We introduce the approximations

\begin{align}
\unp &\approx \sumjn c_j^{n+1}\Phi_j \\
\un &\approx \sumjn c_j^{n}\Phi_j \\
\unm &\approx \sumjn c_j^{n-1}\Phi_j \\
\mathbf{b}^n &\approx \sumjn b_j^{n}\Phi_j
\end{align}

where $\Phi_j = (\phi_1,...,\phi_d)^T$ are prescribed basis functions, and $c_j^{n}$ and $b_j^{n}$ are coefficients to be determined. Let $\mathbf{v} = \Phi_i$.

\begin{equation}
\io\unp \cdot \mathbf{v} dx = \io\left(\sumjn c_j^{n+1}\Phi_j\right)\cdot\Phi_i dx = \sumjn\left(\io \Phi_i\cdot\Phi_j dx\right)c_j^{n+1}
\end{equation}

\begin{align}
\io (2\un - \unm  + \dtt \mathbf{b}^n) \cdot \mathbf{v} dx &= \io \left(2 \sumjn c_j^{n}\Phi_j - \sumjn c_j^{n-1}\Phi_j + \dtt \sumjn b_j^{n}\Phi_j \right) \cdot\Phi_i dx\\
&= 2 \sumjn \left(\io \Phi_i \cdot\Phi_j dx\right)c_j^{n} - \sumjn \left(\io \Phi_i\cdot \Phi_j dx\right)c_j^{n-1} + \dtt \sumjn \left(\io \Phi_i \cdot\Phi_j dx\right)b_j^{n}
\end{align}

\begin{align*}
\io \sigma^n : \nabla \mathbf{v} dx &= \io \sigma(\sumjn c_j^{n}\Phi_j) : \nabla \Phi_i dx \\
&= \sumjn \left(\io \sigma(\Phi_j) : \nabla\Phi_i dx\right)c_j^{n}
\end{align*}

We define the matrices $M$ and $K$, with elements $M_{i,j} = \io \Phi_i \cdot \Phi_j dx$ and $K_{i,j} = \io \sigma(\Phi_j) : \nabla\Phi_i dx$. Since the coefficients $c^{n}$ are the nodal values of $\mathbf{u}$ at time $t=t^n$, we get the linear system

\begin{align}
M\mathbf{u}^{n+1} &= 2 M\mathbf{u}^{n} - M\mathbf{u}^{n-1} - \frac{\dtt}{\rho}K\mathbf{u}^{n} + \dtt M\mathbf{b}^n   \\
&= \left(2 M - \frac{\dtt}{\rho}K\right)\mathbf{u}^n - M\mathbf{u}^{n-1} + \dtt M\mathbf{b}^n
\end{align}

where $\mathbf{u}^{n}$ and $\mathbf{u}^{n-1}$ are known.



\end{document}